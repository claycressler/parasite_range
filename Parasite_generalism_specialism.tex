\documentclass[11pt,reqno,final,pdftex]{amsart}\usepackage[]{graphicx}\usepackage[]{color}
%% maxwidth is the original width if it is less than linewidth
%% otherwise use linewidth (to make sure the graphics do not exceed the margin)
\makeatletter
\def\maxwidth{ %
  \ifdim\Gin@nat@width>\linewidth
    \linewidth
  \else
    \Gin@nat@width
  \fi
}
\makeatother

\definecolor{fgcolor}{rgb}{0.345, 0.345, 0.345}
\newcommand{\hlnum}[1]{\textcolor[rgb]{0.686,0.059,0.569}{#1}}%
\newcommand{\hlstr}[1]{\textcolor[rgb]{0.192,0.494,0.8}{#1}}%
\newcommand{\hlcom}[1]{\textcolor[rgb]{0.678,0.584,0.686}{\textit{#1}}}%
\newcommand{\hlopt}[1]{\textcolor[rgb]{0,0,0}{#1}}%
\newcommand{\hlstd}[1]{\textcolor[rgb]{0.345,0.345,0.345}{#1}}%
\newcommand{\hlkwa}[1]{\textcolor[rgb]{0.161,0.373,0.58}{\textbf{#1}}}%
\newcommand{\hlkwb}[1]{\textcolor[rgb]{0.69,0.353,0.396}{#1}}%
\newcommand{\hlkwc}[1]{\textcolor[rgb]{0.333,0.667,0.333}{#1}}%
\newcommand{\hlkwd}[1]{\textcolor[rgb]{0.737,0.353,0.396}{\textbf{#1}}}%

\usepackage{framed}
\makeatletter
\newenvironment{kframe}{%
 \def\at@end@of@kframe{}%
 \ifinner\ifhmode%
  \def\at@end@of@kframe{\end{minipage}}%
  \begin{minipage}{\columnwidth}%
 \fi\fi%
 \def\FrameCommand##1{\hskip\@totalleftmargin \hskip-\fboxsep
 \colorbox{shadecolor}{##1}\hskip-\fboxsep
     % There is no \\@totalrightmargin, so:
     \hskip-\linewidth \hskip-\@totalleftmargin \hskip\columnwidth}%
 \MakeFramed {\advance\hsize-\width
   \@totalleftmargin\z@ \linewidth\hsize
   \@setminipage}}%
 {\par\unskip\endMakeFramed%
 \at@end@of@kframe}
\makeatother

\definecolor{shadecolor}{rgb}{.97, .97, .97}
\definecolor{messagecolor}{rgb}{0, 0, 0}
\definecolor{warningcolor}{rgb}{1, 0, 1}
\definecolor{errorcolor}{rgb}{1, 0, 0}
\newenvironment{knitrout}{}{} % an empty environment to be redefined in TeX

\usepackage{alltt}
%% DO NOT DELETE OR CHANGE THE FOLLOWING TWO LINES!
%% $Revision$
%% $Date$
\usepackage[round,sort,elide]{natbib}
\usepackage{graphicx}
\usepackage{times}
\usepackage{rotating}
\usepackage{subfig}
\usepackage{color}
\usepackage{verbatim}
\newcommand{\aak}[1]{\textcolor{cyan}{#1}}
\newcommand{\mab}[1]{\textcolor{red}{#1}}
\newcommand{\cec}[1]{\textcolor{blue}{#1}}

\setlength{\textwidth}{6.25in}
\setlength{\textheight}{8.75in}
\setlength{\evensidemargin}{0in}
\setlength{\oddsidemargin}{0in}
\setlength{\topmargin}{-.35in}
\setlength{\parskip}{.1in}
\setlength{\parindent}{0.3in}

%% cleveref must be last loaded package
\usepackage[sort&compress]{cleveref}
\crefname{figure}{Fig.}{Figs.}
\Crefname{figure}{Fig.}{Figs.}
\crefname{table}{Table}{Tables}
\Crefname{table}{Tab.}{Tables}
\crefname{equation}{Eq.}{Eqs.}
\Crefname{equation}{Eq.}{Eqs.}
\crefname{appendix}{Appendix}{Appendices}
\Crefname{appendix}{Appendix}{Appendices}
\creflabelformat{equation}{#2#1#3}
\newcommand{\crefrangeconjunction}{--}
\newcommand{\creflastconjunction}{, and~}

\theoremstyle{plain}
\newtheorem{thm}{Theorem}
\newtheorem{corol}[thm]{Corollary}
\newtheorem{prop}[thm]{Proposition}
\newtheorem{lemma}[thm]{Lemma}
\newtheorem{defn}[thm]{Definition}
\newtheorem{hyp}[thm]{Hypothesis}
\newtheorem{example}[thm]{Example}
\newtheorem{conj}[thm]{Conjecture}
\newtheorem{algorithm}[thm]{Algorithm}
\newtheorem{remark}{Remark}
\renewcommand\thethm{\arabic{thm}}
\renewcommand{\theremark}{}

\numberwithin{equation}{part}
\renewcommand\theequation{\arabic{equation}}
\renewcommand\thesection{\arabic{section}}
\renewcommand\thesubsection{\thesection.\arabic{subsection}}
\renewcommand\thefigure{\arabic{figure}}
\renewcommand\thetable{\arabic{table}}
\renewcommand\thefootnote{\arabic{footnote}}

\newcommand\scinot[2]{$#1 \times 10^{#2}$}
\newcommand{\code}[1]{\texttt{#1}}
\newcommand{\pkg}[1]{\textsf{#1}}
\newcommand{\dlta}[1]{{\Delta}{#1}}
\newcommand{\Prob}[1]{\mathbb{P}\left[#1\right]}
\newcommand{\Expect}[1]{\mathbb{E}\left[#1\right]}
\newcommand{\Var}[1]{\mathrm{Var}\left[#1\right]}
\newcommand{\dd}[1]{\mathrm{d}{#1}}
\newcommand{\citetpos}[1]{\citeauthor{#1}'s \citeyearpar{#1}}
\IfFileExists{upquote.sty}{\usepackage{upquote}}{}
\begin{document}



The purpose of this document is to outline some models that could potentially be used to study the factors that push parasites towards generalism or specialism.
I will propose two approaches to this problem, an evolutionary approach based on game theory and an ecological approach based on optimal foraging theory.
I will discuss the evolutionary approach first, as it was the one I originally suggested in Wales.

\section{Evolution of parasite preference}
The approach that I suggest is fairly simple: for any set of ecological conditions (where ``ecological conditions'' is used loosely to refer to any change in model parameters that might shift the relative costs and benefits of generalism/specialism), what is the evolutionarily stable transmission strategy?

Consider an environmentally transmitted parasite that can potentially transmit to two hosts.
For simplicity of formulation, I will assume that infections, once acquired, are lifelong.
The two host populations are subdivided into susceptible ($S_1$, $S_2$) and
infected ($I_1$, $I_2$) classes.
The parasite ($P$) is transmitted through the environment by parasites shed from infected hosts of each species.
A simple model for the dynamics is:
\begin{align*}
\frac{dS_1}{dt} &= r_1 N_1 \left(\frac{1-N_1}{K_1}\right) - \beta_1 S_1 P \\
\frac{dS_2}{dt} &= r_2 N_2 \left(\frac{1-N_2}{K_2}\right) - \beta_2 S_2 P \\
\frac{dI_1}{dt} &= \beta_1 S_1 P - (\mu_1 + \nu_1) I_1 \\
\frac{dI_2}{dt} &= \beta_2 S_2 P - (\mu_2 + \nu_2) I_2 \\
\frac{dP}{dt} &= \lambda_1 I_1 + \lambda_2 I_2 - \delta P,
\end{align*}
where $N_i = S_i + I_i$.
In the absence of infection, both host populations grow logistically at per-capita rates $r_1$ and $r_2$ to the carrying capacities $K_1$ and $K_2$, respectively.
Both susceptible and infected individuals are assumed to have equal birth rates, and all individuals are born susceptible.
The infection rates per parasite in the environment are $\beta_1$ and $\beta_2$; note that we assume that all parasites in the environment are equivalent (that is, it doesn't matter whether the parasite was shed from host 1 or host 2).
I also assume that $\beta_1$ and $\beta_2$ are determined by the host, not the parasite - the parasite is passive in the environment.
In some circumstances, it might make more sense to assume that parasites shed from host 1 have a higher transmission rate in host 1 (if, for example, the two hosts have slightly different environmental preferences; Fenton et al. 2015).
Infected hosts suffer mortality due to natural causes at the rates $\mu_1$ and $\mu_2$ and due to infection at the rates $\nu_1$ and $\nu_2$.
Infected hosts shed parasites into the environment at the rates $\lambda_1$ and $\lambda_2$, and these are lost at the rate $\delta$.

The $R_0$ for a parasite invading a fully susceptible host community will be (Dobson 2004, Fenton et al. 2015):
\begin{equation}
R_0 = \sqrt{\frac{\lambda_1 \beta_1 K_1}{\gamma(\mu_1 + \nu_1)} + \frac{\lambda_2 \beta_2 K_2}{\gamma(\mu_2 + \nu_2)}}.
\end{equation}
If $R_0 > 1$, the parasite can invade.
Note that this expression is equivalent to $\sqrt{R_{0,1} + R_{0,2}}$, where $R_{0,i}$ is the $R_0$ of a parasite due to transmission only in host $i$.
The implication here is that the parasite could have an $R_0$ less than 1 in each individual host, so long as the sum is greater is greater than 1.
Moreover, this simple model can be extended to consider many host species; in general, the fitness of a parasite infection $n$ host species will be
\begin{equation}
\sqrt{\sum_{i=1}^n R_{0,i}}.
\end{equation}
We can define the ratio $R_{0,i}/R_0$ to evaluate how important each individual host is to the overall fitness of the parasite.
Without any loss of generality, we can assume that the potential hosts are ordered by their $R_{0,i}$ values (i.e., $R_{0,1} > R_{0,2} > \dots > R_{0,n}$): the parasite has the highest single-host $R_0$ in the first host, and the lowest in the $n^{\text{th}}$ host.
These differences in $R_{0,i}$ can be set by a single parameter (e.g., holding all parameters constant across hosts except $K_i$ would produce this pattern) or by covarying multiple parameters.
The choice of how to vary the single-host $R_0$ could potentially be motivated by how we choose to implement any evolutionary trade-offs.

A similar analysis can be used to determine whether a mutant parasite that parasitizes a different number of hosts can invade.
To make this concrete, consider that there is a community of three hosts; the resident parasite infects two of these hosts.
A mutant parasite that infects all three hosts can invade if
\begin{equation}\label{eq:invasion}
\sqrt{\frac{\lambda_1 \beta_1 \hat{S}_1}{\gamma(\mu_1 + \nu_1)} + \frac{\lambda_2 \beta_2 \hat{S}_2}{\gamma(\mu_2 + \nu_2)} + \frac{\lambda_3 \beta_3 K_3}{\gamma(\mu_3 + \nu_3)}} > 1,
\end{equation}
where $\hat{S}_1$ and $\hat{S}_2$ are the equilibrium abundances of susceptible hosts, set by the resident parasite.

The key to making this analysis interesting are the potential trade-offs between traits that might affect the number of hosts utilized by the parasite.
In particular, let's define $\eta$ to be the number of hosts the parasite utilizes.
Effects of $\eta$ on other parameters of the model that might be interesting to consider are:
\begin{enumerate}
\item a trade-off between $\eta$ and $\lambda_i$: the more hosts the parasite infects, the less its shedding rate in each host;
\item a trade-off between $\eta$ and $\nu_i$: the more hosts the parasite infects, the greater its virulence in each host;
\item a trade-off between $\eta$ and $\gamma$: the more hosts a parasite infects, the less hardy each parasite is in the environment;
\item a trade-off between $\eta$ and $\beta_i$: the more hosts a parasite infects, the less transmissible each parasite is.
\end{enumerate}
Any of these trade-offs might be interesting to explore, as each seems biologically reasonable: the first essentially posits that transmitting to many hosts compromises within-host replication ability; the third and fourth essentially posit that transmitting to many hosts compromises the quality or viability of parasites in the environment; the second is probably the most tenuous, positing that transmitting to many hosts makes the parasite more harmful to each one.
More complicated relationships could also be considered: for example, if increasing the number of infected hosts decreases within-host replication, then increasing $\eta$ might simultaneously decrease $\lambda_i$ and $\nu_i$ (a classic transmission--virulence trade-off controlled by the parasite's generality).

This evolutionary model can help us make predictions about how ecological conditions might affect the evolution of generality.
For example, the invasion criterion (Eq. \ref{eq:invasion}) can be rearranged to yield
\begin{equation}
\sqrt{\frac{\lambda_1 \beta_1 \hat{S}_1}{\mu_1 + \nu_1} + \frac{\lambda_2 \beta_2 \hat{S}_2}{\mu_2 + \nu_2} + \frac{\lambda_3 \beta_3 K_3}{\mu_3 + \nu_3}} > \sqrt{\gamma}.
\end{equation}
Thus, the higher the mortality rate in the environment, the harder it will be for a mutant parasite that infects more hosts to invade.
This suggests that parasites living in high-mortality environments might be more likely to be specialists.

\section{Ecology of parasite preference}
The ecological approach is based on classic optimal foraging theory (Charnov 1976), which deals with the question of diet breadth of a predator.
There are many different prey items that could potentially be included in the diet of a predator, $i = 1, \dots, N$.
Each prey item has a profitability to the predator of $E_i$ (which can be thought of as the energy content of the prey).
The search time (or, inversely, the encounter rate) for prey type $i$ is $s_i$.
The handling time for prey type $i$ is $h_i$.
The prey types can be ordered in decreasing profitability, so $E_1/h_1 > E_2/h_2 > E_3/h_3 > \dots > E_N/h_N$.
Assume that the predator currently predates on prey types 1 thru $n$ - should it include the $(n+1)^{\text{th}}$ prey item?
Optimal foraging theory says that this prey item will be included if
\begin{equation}\label{eq:OFT}
\frac{E_{n+1}}{h_{n+1}} > \frac{\bar{E}}{\bar{h} + \bar{s}},
\end{equation}
where $\bar{E}$ is the average profitability of items already in the diet, $\bar{h}$ is the average handling time of items already in the diet, and $\bar{s}$ is the average search time to find an item in the diet.
The intuition behind this equation is as follows: imagine that the predator has encountered an individual of type $n+1$: should it eat it or keep searching for other prey items?
It should it eat it only if the rate it gains energy from that prey item (the profitability divided by the time to handle it) is greater than the average rate it gains energy from searching for and handling all other items in the diet.
A note to the particularly savvy: I realize I am playing a bit fast and loose with the analytical form of the optimal foraging prediction equation.
\begin{comment}
What I am showing is essentially the optimal foraging theory prediction if a predator
The form of the equation I have shown is essentially the per-prey-item prediction.
However, because optimal foraging theory deals with predators, a single predator will potentially encounter many prey in a given period of time, so technically the prediction should include information about how many prey of each type will be encountered.
A parasite, however, needs only to encounter a single host (which it will infect, or not), so the per-prey-item form makes sense.
\end{comment}

It remains to relate these to parameters to the situation of an environmentally tranmsitted parasite.
Although this particular question has not been tackled, other authors have considered applying optimal foraging theory to parasitoids (Iwasa et al. 1984, Charnov and Stephens 1988) and lytic phages (Guyader and Burch 2008).
The following is my first attempt at making this relation.

To start as simply as possible, let's assume that the profitability depends on the shedding rate and the duration of infection, the search time depends on the transmission (encounter) rate and the abundance of the host (which we will assume is constant), and the handling time depends on the development rate of the parasite (assuming that eggs are ingested from the environment and must develop from larvae into adults before shedding begins).
Let $\lambda_i$ be the shedding rate from host $i$; let $\mu_i$ be the mortality rate of host $i$; let $\beta_i$ be the encounter rate with host $i$; let $H_i$ be the abundance of host $i$; and let $\sigma_i$ be the development rate of the parasite.
Then
\begin{align}
E_i &= \frac{\lambda_i}{\mu_i}, \\
s_i &= \frac{1}{\beta_i H_i}, \\
h_i &= \frac{1}{\sigma_i}.
\end{align}

Under this very simple model, the $(n+1)^{\text{th}}$ host will be included if (refer to \ref{eq:OFT}):
\begin{equation}
\frac{\lambda_{n+1} \sigma_{n+1}}{\mu_{n+1}} > \frac{\sum_{i=1}^n \frac{\lambda_i}{\mu_i}}{\frac{1}{\sum_{i=1}^n \beta_i H_i} + \frac{1}{\sum_{i=1}^n \sigma_i}}.
\end{equation}
Even under this simple model, however, we can begin to make some predictions, and what I like about this model is that I feel like it might be easier to relate these predictions back to the traits of the parasite.
For example, parasites with long development rates (large $\sigma_i$: larger-bodied parasites, perhaps?) are more likely to be generalists.
Parasites infecting short-lived hosts, on the other hand (large $\mu_i$), are likely to be specialists.
The greater then number of hosts around, the more likely a parasite is to be a specialist, so parasites inhabiting eutrophic environments may be more likely to be specialists.

Moreover, this model can be extended in couple of interesting ways.
For example, we could assume that the parasite can only infect uninfected hosts, in which case we might multiply $H_i$ by $1-P_i$, where $P_i$ is the prevalence of infection in host $i$.
This would affect the likely of specialism vs generalism, because if the prevalence of infection is very high, generalism becomes much more likely.
We could extend this idea a bit further by considering competitive dominance of the parasite: perhaps $P_i$ is not the prevalence of infection by the focal parasite in host $i$, but is instead the prevalence of competitively-dominant parasites.
In such a case, we would predict that parasites that are competitively inferior should be generalists.

We can also extend the model to consider mortality.
For example, the parasite might die while searching for a host (the parameter $\gamma$ in the evolutionary model).
The parasite might also die while ``handling'' the host (developing from egg to adult).
In particular, we might want to consider that parasites are especially senstive to being cleared by the host immune system during this period (or displaced by a competitive dominant).
Adding mortality to the model is slightly more complicated, since the basic theory developed for parasitoids and predators doesn't quite apply.
I am still working on adding mortality to the model, but I wanted to outline the basic optimal foraging theory approach to solicit feedback on it.
In particular, \textbf{what parasite traits and processes will affect the search time, handling time, and profitability of each host type?}

\bibliographystyle{ecology}
\bibliography{library.bib}

\end{document}
